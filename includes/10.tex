\section{Überblick}

\begin{frame}
  {Verbtypen}
\end{frame}

\section{Objekte und Valenz}

\begin{frame}
  {Direkte Objekte}
  \pause
   Kaum anders als beim Subjekt.
  \begin{itemize}[<+->]
    \item \alert{Akkusativ-Ergänzungen zum Verb}
    \item \alert{oder Nebensätze an deren Stelle}
  \end{itemize}
  \pause
  \Halbzeile
  Und Doppelakkusative?\\
  \begin{exe}
    \ex\label{ex:akkusativeunddirekteobjekte158}
    \begin{xlist}
      \ex[ ]{\label{ex:akkusativeunddirekteobjekte159} Ich lehre \alert{ihn} \orongsch{das Schwimmen}.}
      \pause
      \ex[*]{\label{ex:akkusativeunddirekteobjekte160} \orongsch{Das Schwimmen} wird \alert{ihn} gelehrt.}
      \pause
      \ex[*]{\label{ex:akkusativeunddirekteobjekte161} \alert{Er} wird \orongsch{das Schwimmen} gelehrt.}
      \pause
      \ex[ ]{\label{ex:akkusativeunddirekteobjekte161} Hier wird \orongsch{das Schwimmen} gelehrt.}
    \end{xlist}
  \end{exe}
  \pause
  \begin{itemize}[<+->]
    \item unterschiedlicher Status der Akkusativ-Ergänzungen
    \item Die "`erste"' entspricht der normaler Transitiva.
    \item \grau{Korrektur zum Buch: Doppelakkusative bilden unpersönliche Passive.}
  \end{itemize} 
\end{frame}

\begin{frame}
  {Indirekte Objekte}
  \pause
  Welche Dative sind Ergänzungen (= Teil der Valenz)?\\
  \pause
  \Halbzeile
  \begin{exe}
    \ex\label{ex:dativeundindirekteobjekte166}
    \begin{xlist}
      \ex[ ]{\label{ex:dativeundindirekteobjekte167} \alert{Alma} gibt \gruen{ihm} heute ein Buch.}
      \pause
      \ex[ ]{\label{ex:dativeundindirekteobjekte168} \alert{Alma} fährt \gruen{mir} heute aber wieder schnell.}
      \pause
      \ex[ ]{\label{ex:dativeundindirekteobjekte169} \alert{Alma} mäht \gruen{mir} heute den Rasen.}
      \pause
      \ex[ ]{\label{ex:dativeundindirekteobjekte170} \alert{Alma} klopft \gruen{mir} heute auf die Schulter.}
    \end{xlist}
  \end{exe}
  \Halbzeile
  \pause
  Recht einfache Entscheidung, da wir Passiv\\
  als \alert{Valenzänderung} beschreiben:\\
  \pause
  \begin{exe}
    \ex\label{ex:dativeundindirekteobjekte171}
    \begin{xlist}
      \ex[ ]{\label{ex:dativeundindirekteobjekte172} \gruen{Er} bekommt \alert{von Alma} heute ein Buch gegeben.}
      \ex[*]{\label{ex:dativeundindirekteobjekte173} \rot{Ich} bekomme \alert{von Alma} heute aber wieder schnell gefahren.}
      \ex[ ]{\label{ex:dativeundindirekteobjekte174} \gruen{Ich} bekomme \alert{von Alma} heute den Rasen gemäht.}
      \ex[ ]{\label{ex:dativeundindirekteobjekte175} \gruen{Ich} bekomme \alert{von Alma} heute auf die Schulter geklopft.}
    \end{xlist}
  \end{exe}
\end{frame}

\begin{frame}
  {Die vier wichtigen verbabhängigen Dative}
  \pause
  \begin{exe}
    \ex\label{ex:dativeundindirekteobjekte166x}
    \begin{xlist}
      \ex{\label{ex:dativeundindirekteobjekte167x} Alma gibt \gruen{ihm} heute ein Buch.}
      \pause
      \ex{\label{ex:dativeundindirekteobjekte168x} Alma fährt \orongsch{mir} heute aber wieder schnell.}
      \pause
      \ex{\label{ex:dativeundindirekteobjekte169x} Alma mäht \alert{mir} heute den Rasen.}
      \pause
      \ex{\label{ex:dativeundindirekteobjekte170x} Alma klopft \alert{mir} heute auf die Schulter.}
    \end{xlist}
  \end{exe}
  \Halbzeile
  \pause
  \begin{itemize}[<+->]
    \item (\ref{ex:dativeundindirekteobjekte167x}) = \gruen{Ergänzung} bei ditransitivem Verb
    \item (\ref{ex:dativeundindirekteobjekte168x}) = \orongsch{Bewertungsdativ} (Angabe, im Vorfeld\slash direkt nach finitem Verb)
    \item (\ref{ex:dativeundindirekteobjekte169x}) = \alert{Nutznießerdativ} (\alert{Ergänzung per Valenzerweiterung})
    \item (\ref{ex:dativeundindirekteobjekte170x}) = \alert{Pertinenzdativ} (\alert{Ergänzung per Valenzerweiterung})
      \Halbzeile
    \item Bewertungsdativ, Nutznießerdativ und Pertinenzdativ\\
      nennt man auch \textit{freie Dative}.
  \end{itemize}
\end{frame}

\begin{frame}
  {Valenzveränderungen im Beispiel}
  \pause
  1.~Wir beginnen mit einem Verb mit \alert{Nom\_Ag} und einem \orongsch{Akk}:\\
  \pause
  \Halbzeile
  \begin{exe}
    \ex \alert{Alma} mäht \orongsch{den Rasen}.
  \end{exe}
  \Zeile
  \pause
  2.~Der \gruen{Nutznießerdativ} wird als Valenzerweiterung hinzugefügt:\\
  \pause
  \Halbzeile
  \begin{exe}
    \ex \alert{Alma} mäht \gruen{meinem Kollegen} \orongsch{den Rasen}.
  \end{exe}
  \Zeile
  \pause
  3.~Das Rezipientenpassiv (Valenzänderung) kann jetzt gebildet werden:
  \pause
  \Halbzeile
  \begin{exe}
    \ex \gruen{Mein Kollege} bekommt \alert{(von Alma)} \orongsch{den Rasen} gemäht.
  \end{exe}
\end{frame}

\begin{frame}
  {Präpositionalobjekte}
  \pause
  PP-Angabe vs.\ PP-Ergänzung: oft schwierig zu entscheiden.\\
  \Viertelzeile
  \pause
  \begin{exe}
    \ex\label{ex:ppergaenzungenundppangaben189}
    \begin{xlist}
      \ex{\label{ex:ppergaenzungenundppangaben190} Viele Menschen leiden \alert{unter Vorurteilen}.}
      \pause
      \ex{\label{ex:ppergaenzungenundppangaben191} Viele Menschen schwitzen \orongsch{unter Sonnenschirmen}.}
    \end{xlist}
  \end{exe}
  \Viertelzeile
  \pause
  \begin{itemize}[<+->]
    \item \alert{Ergänzungen}:
      \begin{itemize}[<+->]
        \item Semantik der PP nur verbgebunden interpretierbar
        \item = semantische Rolle der PP vom Verb zugewiesen
      \end{itemize}
    \item \orongsch{Angaben}:
      \begin{itemize}[<+->]
        \item Semantik der PP selbständig erschließbar (lokal unter)
        \item = "`semantische Rolle"' der PP von der Präposition zugewiesen
      \end{itemize}
      \Viertelzeile
    \item \alert{Sehen Sie, wie schnell man in der (Grund-)Schulgrammatik\\
      in gefährliche linguistische Fahrwasser gerät?}
    \item \rot{Wenn Sie dieses Wissen nicht haben, unterrichten Sie sehr leicht\\
      komplett Falsches, zumal wenn es im Lehrbuch falsch steht.}
  \end{itemize}
\end{frame}


\begin{frame}
  {Der umstrittene PP-Angaben-Test}
  \pause
  Die PP mit \textit{"`Dies geschieht PP."'} aus dem Satz auskoppeln.\\
  \Halbzeile
  \pause
  \begin{exe}
    \ex\label{ex:ppergaenzungenundppangaben192}
    \begin{xlist}
      \ex[*]{\label{ex:ppergaenzungenundppangaben193} Viele Menschen leiden.
      \rot{Dies geschieht unter Vorurteilen.}}
        \pause
      \ex[ ]{\label{ex:ppergaenzungenundppangaben194} Viele Menschen schwitzen.
      \alert{Dies geschieht unter Sonnenschirmen.}}
        \pause
      \ex[*]{\label{ex:ppergaenzungenundppangaben195} Mausi schickt einen Brief.
      \rot{Dies geschieht an ihre Mutter.}}
        \pause
      \ex[*]{\label{ex:ppergaenzungenundppangaben196} Mausi befindet sich.
      \rot{Dies geschieht in Hamburg.}}
        \pause
      \ex[?]{\label{ex:ppergaenzungenundppangaben197} Mausi liegt.
      \orongsch{Dies geschieht auf dem Bett.}}
    \end{xlist}
  \end{exe}
  \Halbzeile
  \pause
  \begin{itemize}[<+->]
    \item der beste Test, den es gibt
    \item trotz Problemen
    \item \rot{Verlangen Sie von Schüler*innen keine Entscheidungen,\\
    die Sie selber nicht operationalisieren können!}
  \end{itemize}
\end{frame}

\section{Übung}

\begin{frame}
  {Verbklassen}
\end{frame}

\section{Ausblick}

\begin{frame}
  {Lexikalische Semantik}
  \onslide<+->
  \begin{itemize}[<+->]
    \item \ldots
  \end{itemize}
\end{frame}
