\section{Überblick}

\section{Passive}

\begin{frame}
  {\textit{werden}-Passiv oder Vorgangspassiv}
  \pause
  "`Nur transitive Verben können passiviert werden."'\pause\rot{--- Nein!}
  \pause
    \begin{exe}
    \addtolength\itemsep{-0.25\baselineskip}
      \ex\label{ex:werdenpassivundverbtypen110}
      \begin{xlist}\addtolength\itemsep{-0.5\baselineskip}
          \ex[ ]{\label{ex:werdenpassivundverbtypen111} \alert{Johan} wäscht \orongsch{den Wagen}.}
          \ex[ ]{\label{ex:werdenpassivundverbtypen112} \orongsch{Der Wagen} wird \alert{(von Johan)} gewaschen.}
      \end{xlist}
      \pause
      \ex\label{ex:werdenpassivundverbtypen113}
      \begin{xlist}\addtolength\itemsep{-0.5\baselineskip}
          \ex[ ]{\label{ex:werdenpassivundverbtypen114} \alert{Alma} schenkt \gruen{dem Schlossherrn} \orongsch{den Roman}.}
          \ex[ ]{\label{ex:werdenpassivundverbtypen115} \orongsch{Der Roman} wird \gruen{dem Schlossherrn} \alert{(von Alma)} geschenkt.}
      \end{xlist}
      \pause
      \ex\label{ex:werdenpassivundverbtypen116}
      \begin{xlist}\addtolength\itemsep{-0.5\baselineskip}
          \ex[ ]{\label{ex:werdenpassivundverbtypen117} \alert{Johan} bringt \orongsch{den Brief} zur Post.}
          \ex[ ]{\label{ex:werdenpassivundverbtypen118} \orongsch{Der Brief} wird \alert{(von Johan)} zur Post gebracht.}
      \end{xlist}
      \pause
      \ex\label{ex:werdenpassivundverbtypen119}
      \begin{xlist}\addtolength\itemsep{-0.5\baselineskip}
          \ex[ ]{\label{ex:werdenpassivundverbtypen120} \alert{Der Maler} dankt \gruen{den Fremden}.}
          \ex[ ]{\label{ex:werdenpassivundverbtypen121} \gruen{Den Fremden} wird \alert{(vom Maler)} gedankt.}
      \end{xlist}
      \pause
      \ex\label{ex:werdenpassivundverbtypen122}
      \begin{xlist}\addtolength\itemsep{-0.5\baselineskip}
          \ex[ ]{\label{ex:werdenpassivundverbtypen123} \alert{Johan} arbeitet hier immer montags.}
          \ex[ ]{\label{ex:werdenpassivundverbtypen124} Montags wird hier \alert{(von Johan)} immer gearbeitet.}
      \end{xlist}
      \pause
      \ex\label{ex:werdenpassivundverbtypen125}
      \begin{xlist}\addtolength\itemsep{-0.5\baselineskip}
          \ex[ ]{\label{ex:werdenpassivundverbtypen126} \alert{Der Ball} platzt bei zu hohem Druck.}
          \ex[*]{\label{ex:werdenpassivundverbtypen127} Bei zu hohem Druck wird \rot{(vom Ball)} geplatzt.}
      \end{xlist}
      \pause
      \ex\label{ex:werdenpassivundverbtypen128}
      \begin{xlist}\addtolength\itemsep{-0.5\baselineskip}
          \ex[ ]{\label{ex:werdenpassivundverbtypen129} \alert{Der Rottweiler} fällt \gruen{Michelle} auf.}
          \ex[*]{\label{ex:werdenpassivundverbtypen130} \alert{Michelle} wird \rot{(von dem Rottweiler)} aufgefallen.}
      \end{xlist}
    \end{exe}
\end{frame}

\begin{frame}
  {Was passiert beim Vorgangspassiv?}
  \pause
  \begin{itemize}[<+->]
    \item Auxiliar: \textit{werden}, Verbform: Partizip
    \item für Passivierbarkeit relevant: \alert{die Nominativ-Ergänzung!}
      \Halbzeile
    \item \alert{Passivierung = Valenzänderung}:
      \begin{itemize}[<+->]
        \item Nominativ-Ergänzung → optionale \textit{von}-PP-Angabe
        \item eventuelle Akkusativ-Ergänzung → obligatorische Nominativ-Ergänzung
        \item kein Akkusativ: kein "`Subjekt"' = keine Nom-Erg (\textit{es} ist positional)
        \item \grau{Dativ-Ergänzung → Dativ-Ergänzung (usw.)}
        \item \grau{Angaben: keine Änderung}
      \end{itemize}
    \Halbzeile
  \item \alert{nicht passivierbare Verben}?
    \begin{itemize}[<+->]
      \item {ohne }\rot{agentivische}\alert{ Nominativ-Ergänzung}
      \item Achtung! Gilt nur mit prototypischem Charakter\ldots
      \item Siehe Vertiefung 14.2 auf S.~439!
    \end{itemize}
  \end{itemize}
\end{frame}

\begin{frame}
  {Feinere Klassifikation von Verben}
  \pause
  \begin{itemize}[<+->]
    \item Neuklassifikation vor dem Hintergrund des Vorgangspassivs
    \item Wenn so eine Klassifikation einen Wert haben soll:\\
      \alert{Berücksichtigung der semantischen Rollen unabdinglich!}
    \item Bedingung für Vorgangs-Passiv: \alert{Nom\_Ag}
  \end{itemize} 
  \pause
  \Zeile
  \centering
  \scalebox{0.9}{\begin{tabular}{lllll}
    \toprule
    \textbf{Valenz} & \textbf{Passiv} & \textbf{Name} & \textbf{Beispiel} \\
    \midrule
    \alert{Nom\_Ag} & ja & Unergative & \textit{arbeiten} \\
    Nom & nein & Unakkusative & \textit{platzen} \\
    \alert{Nom\_Ag}, Akk & ja & Transitive & \textit{waschen} \\
    \alert{Nom\_Ag}, Dat & ja & unergative Dativverben & \textit{danken} \\
    Nom, Dat & nein & unakkusative Dativverben & \textit{auf"|fallen} \\
    \alert{Nom\_Ag}, Dat, Akk & ja & Ditransitive & \textit{geben} \\
    \bottomrule
  \end{tabular}}\\
  \raggedright
  \Zeile
  \pause
  Immer noch nichts als eine reine Bequemlichkeitsterminologie,\\
  um bestimmte (durchaus wichtige) Valenzmuster hervorzuheben.
\end{frame}


\begin{frame}
  {\textit{bekommen}-Passiv oder Rezipientenpassiv}
  \pause
  Es gibt nicht "`das Passiv im Deutschen"'.\\
  \Halbzeile
  \pause
  \begin{exe}
    \ex\label{ex:bekommenpassiv138}
    \begin{xlist}
      \ex[ ]{\small\label{ex:bekommenpassiv139} \gruen{Mein Kollege} bekommt \orongsch{den Wagen} \alert{(von Johan)} gewaschen.}
      \pause
      \ex[ ]{\small\label{ex:bekommenpassiv140} \gruen{Der Schlossherr} bekommt \orongsch{den Roman} \alert{(von Alma)} geschenkt.}
      \pause
      \ex[ ]{\small\label{ex:bekommenpassiv141} \gruen{Mein Kollege} bekommt \orongsch{den Brief} \alert{(von Johan)} zur Post gebracht.}
      \pause
      \ex[ ]{\small\label{ex:bekommenpassiv142} \gruen{Die Fremden} bekommen \alert{(von dem Maler)} gedankt.}
      \pause
      \ex[?]{\small\label{ex:bekommenpassiv143} \gruen{Mein Kollege} bekommt hier immer montags \alert{(von Johan)} gearbeitet.}
      \pause
      \ex[*]{\small\label{ex:bekommenpassiv144} \gruen{Mein Kollege} bekommt bei zu hohem Druck \rot{(von dem Ball)} geplatzt.}
      \pause
      \ex[*]{\small\label{ex:bekommenpassiv145} \gruen{Michelle} bekommt \rot{(von dem Rottweiler)} aufgefallen.}
    \end{xlist}
  \end{exe}
  \pause\Halbzeile
  \alert{Das ist eine Passivbildung, die genauso den Nom\_Ag betrifft\\
  wie das Vorgangspassiv.}
\end{frame}

\begin{frame}
  {Was passiert beim Rezipientenpassiv?}
  \pause
  Alles, was sich verglichen mit Vorgangspassiv nicht unterscheidet, grau.\\
  \Halbzeile
  \pause
  \begin{itemize}[<+->]
    \item Auxiliar: \textit{bekommen} (evtl.\ \textit{kriegen}), \grau{Verbform: Partizip}
    \item \grau{für Passivierbarkeit relevant: die Nominativ-Ergänzung!}
      \Halbzeile
    \item \grau{Passivierung = Valenzänderung}:
      \begin{itemize}[<+->]
        \item \grau{Nominativ-Ergänzung → optionale \textit{von}-PP-Angabe}
        \item eventuelle Akkusativ-Ergänzung: → Akkusativ-Ergänzung
        \item \alert{Dativ-Ergänzung → Nominativ-Ergänzung}
        \item \rot{kein Dativ: kein Rezipientenpassiv}
        \item \grau{Angaben: keine Änderung}
      \end{itemize}
    \Halbzeile
  \item \grau{nicht passivierbare Verben?}
    \begin{itemize}[<+->]
      \item \grau{ohne agentivische Nominativ-Ergänzung}
      \item \grau{Achtung! Gilt nur mit prototypischem Charakter\ldots}
      \item \grau{Siehe Vertiefung 14.2 auf S.~439!}
    \end{itemize}
  \end{itemize}
\end{frame}

\begin{frame}
  {Rezipientenpassiv bei unergativen Verben}
  \pause
  Warum war dieser Satz zweifelhaft?\\
  \begin{exe}
    \ex[?]{\small \gruen{Mein Kollege} bekommt hier immer montags \alert{(von Johan)} gearbeitet.}
  \end{exe}
  \pause
  \Halbzeile
  Ist der zugehörige Aktivsatz besser?\\
  \pause
  \begin{exe}
    \ex[?]{\small Montags arbeitet \alert{Johan} \gruen{meinem Kollegen} hier immer.}
  \end{exe}
  \pause
  \begin{itemize}[<+->]
    \item Nein.
    \item \alert{keine Frage des Rezipientenpassivs}
    \item bei diesen Verben: eher \textit{für}-PP
  \end{itemize}
\end{frame}


\section{Objekte und Valenz}

\begin{frame}
  {Direkte Objekte}
  \pause
   Kaum anders als beim Subjekt.
  \begin{itemize}[<+->]
    \item \alert{Akkusativ-Ergänzungen zum Verb}
    \item \alert{oder Nebensätze an deren Stelle}
  \end{itemize}
  \pause
  \Halbzeile
  Und Doppelakkusative?\\
  \begin{exe}
    \ex\label{ex:akkusativeunddirekteobjekte158}
    \begin{xlist}
      \ex[ ]{\label{ex:akkusativeunddirekteobjekte159} Ich lehre \alert{ihn} \orongsch{das Schwimmen}.}
      \pause
      \ex[*]{\label{ex:akkusativeunddirekteobjekte160} \orongsch{Das Schwimmen} wird \alert{ihn} gelehrt.}
      \pause
      \ex[*]{\label{ex:akkusativeunddirekteobjekte161} \alert{Er} wird \orongsch{das Schwimmen} gelehrt.}
      \pause
      \ex[ ]{\label{ex:akkusativeunddirekteobjekte161} Hier wird \orongsch{das Schwimmen} gelehrt.}
    \end{xlist}
  \end{exe}
  \pause
  \begin{itemize}[<+->]
    \item unterschiedlicher Status der Akkusativ-Ergänzungen
    \item Die "`erste"' entspricht der normaler Transitiva.
    \item \grau{Korrektur zum Buch: Doppelakkusative bilden unpersönliche Passive.}
  \end{itemize} 
\end{frame}

\begin{frame}
  {Indirekte Objekte}
  \pause
  Welche Dative sind Ergänzungen (= Teil der Valenz)?\\
  \pause
  \Halbzeile
  \begin{exe}
    \ex\label{ex:dativeundindirekteobjekte166}
    \begin{xlist}
      \ex[ ]{\label{ex:dativeundindirekteobjekte167} \alert{Alma} gibt \gruen{ihm} heute ein Buch.}
      \pause
      \ex[ ]{\label{ex:dativeundindirekteobjekte168} \alert{Alma} fährt \gruen{mir} heute aber wieder schnell.}
      \pause
      \ex[ ]{\label{ex:dativeundindirekteobjekte169} \alert{Alma} mäht \gruen{mir} heute den Rasen.}
      \pause
      \ex[ ]{\label{ex:dativeundindirekteobjekte170} \alert{Alma} klopft \gruen{mir} heute auf die Schulter.}
    \end{xlist}
  \end{exe}
  \Halbzeile
  \pause
  Recht einfache Entscheidung, da wir Passiv\\
  als \alert{Valenzänderung} beschreiben:\\
  \pause
  \begin{exe}
    \ex\label{ex:dativeundindirekteobjekte171}
    \begin{xlist}
      \ex[ ]{\label{ex:dativeundindirekteobjekte172} \gruen{Er} bekommt \alert{von Alma} heute ein Buch gegeben.}
      \ex[*]{\label{ex:dativeundindirekteobjekte173} \rot{Ich} bekomme \alert{von Alma} heute aber wieder schnell gefahren.}
      \ex[ ]{\label{ex:dativeundindirekteobjekte174} \gruen{Ich} bekomme \alert{von Alma} heute den Rasen gemäht.}
      \ex[ ]{\label{ex:dativeundindirekteobjekte175} \gruen{Ich} bekomme \alert{von Alma} heute auf die Schulter geklopft.}
    \end{xlist}
  \end{exe}
\end{frame}

\begin{frame}
  {Die vier wichtigen verbabhängigen Dative}
  \pause
  \begin{exe}
    \ex\label{ex:dativeundindirekteobjekte166x}
    \begin{xlist}
      \ex{\label{ex:dativeundindirekteobjekte167x} Alma gibt \gruen{ihm} heute ein Buch.}
      \pause
      \ex{\label{ex:dativeundindirekteobjekte168x} Alma fährt \orongsch{mir} heute aber wieder schnell.}
      \pause
      \ex{\label{ex:dativeundindirekteobjekte169x} Alma mäht \alert{mir} heute den Rasen.}
      \pause
      \ex{\label{ex:dativeundindirekteobjekte170x} Alma klopft \alert{mir} heute auf die Schulter.}
    \end{xlist}
  \end{exe}
  \Halbzeile
  \pause
  \begin{itemize}[<+->]
    \item (\ref{ex:dativeundindirekteobjekte167x}) = \gruen{Ergänzung} bei ditransitivem Verb
    \item (\ref{ex:dativeundindirekteobjekte168x}) = \orongsch{Bewertungsdativ} (Angabe, im Vorfeld\slash direkt nach finitem Verb)
    \item (\ref{ex:dativeundindirekteobjekte169x}) = \alert{Nutznießerdativ} (\alert{Ergänzung per Valenzerweiterung})
    \item (\ref{ex:dativeundindirekteobjekte170x}) = \alert{Pertinenzdativ} (\alert{Ergänzung per Valenzerweiterung})
      \Halbzeile
    \item Bewertungsdativ, Nutznießerdativ und Pertinenzdativ\\
      nennt man auch \textit{freie Dative}.
  \end{itemize}
\end{frame}

\begin{frame}
  {Valenzveränderungen im Beispiel}
  \pause
  1.~Wir beginnen mit einem Verb mit \alert{Nom\_Ag} und einem \orongsch{Akk}:\\
  \pause
  \Halbzeile
  \begin{exe}
    \ex \alert{Alma} mäht \orongsch{den Rasen}.
  \end{exe}
  \Zeile
  \pause
  2.~Der \gruen{Nutznießerdativ} wird als Valenzerweiterung hinzugefügt:\\
  \pause
  \Halbzeile
  \begin{exe}
    \ex \alert{Alma} mäht \gruen{meinem Kollegen} \orongsch{den Rasen}.
  \end{exe}
  \Zeile
  \pause
  3.~Das Rezipientenpassiv (Valenzänderung) kann jetzt gebildet werden:
  \pause
  \Halbzeile
  \begin{exe}
    \ex \gruen{Mein Kollege} bekommt \alert{(von Alma)} \orongsch{den Rasen} gemäht.
  \end{exe}
\end{frame}

\begin{frame}
  {Präpositionalobjekte}
  \pause
  PP-Angabe vs.\ PP-Ergänzung: oft schwierig zu entscheiden.\\
  \Viertelzeile
  \pause
  \begin{exe}
    \ex\label{ex:ppergaenzungenundppangaben189}
    \begin{xlist}
      \ex{\label{ex:ppergaenzungenundppangaben190} Viele Menschen leiden \alert{unter Vorurteilen}.}
      \pause
      \ex{\label{ex:ppergaenzungenundppangaben191} Viele Menschen schwitzen \orongsch{unter Sonnenschirmen}.}
    \end{xlist}
  \end{exe}
  \Viertelzeile
  \pause
  \begin{itemize}[<+->]
    \item \alert{Ergänzungen}:
      \begin{itemize}[<+->]
        \item Semantik der PP nur verbgebunden interpretierbar
        \item = semantische Rolle der PP vom Verb zugewiesen
      \end{itemize}
    \item \orongsch{Angaben}:
      \begin{itemize}[<+->]
        \item Semantik der PP selbständig erschließbar (lokal unter)
        \item = "`semantische Rolle"' der PP von der Präposition zugewiesen
      \end{itemize}
      \Viertelzeile
    \item \alert{Sehen Sie, wie schnell man in der (Grund-)Schulgrammatik\\
      in gefährliche linguistische Fahrwasser gerät?}
    \item \rot{Wenn Sie dieses Wissen nicht haben, unterrichten Sie sehr leicht\\
      komplett Falsches, zumal wenn es im Lehrbuch falsch steht.}
  \end{itemize}
\end{frame}


\begin{frame}
  {Der umstrittene PP-Angaben-Test}
  \pause
  Die PP mit \textit{"`Dies geschieht PP."'} aus dem Satz auskoppeln.\\
  \Halbzeile
  \pause
  \begin{exe}
    \ex\label{ex:ppergaenzungenundppangaben192}
    \begin{xlist}
      \ex[*]{\label{ex:ppergaenzungenundppangaben193} Viele Menschen leiden.
      \rot{Dies geschieht unter Vorurteilen.}}
        \pause
      \ex[ ]{\label{ex:ppergaenzungenundppangaben194} Viele Menschen schwitzen.
      \alert{Dies geschieht unter Sonnenschirmen.}}
        \pause
      \ex[*]{\label{ex:ppergaenzungenundppangaben195} Mausi schickt einen Brief.
      \rot{Dies geschieht an ihre Mutter.}}
        \pause
      \ex[*]{\label{ex:ppergaenzungenundppangaben196} Mausi befindet sich.
      \rot{Dies geschieht in Hamburg.}}
        \pause
      \ex[?]{\label{ex:ppergaenzungenundppangaben197} Mausi liegt.
      \orongsch{Dies geschieht auf dem Bett.}}
    \end{xlist}
  \end{exe}
  \Halbzeile
  \pause
  \begin{itemize}[<+->]
    \item der beste Test, den es gibt
    \item trotz Problemen
    \item \rot{Verlangen Sie von Schüler*innen keine Entscheidungen,\\
    die Sie selber nicht operationalisieren können!}
  \end{itemize}
\end{frame}



\section{Übung}


\section{Ausblick}

