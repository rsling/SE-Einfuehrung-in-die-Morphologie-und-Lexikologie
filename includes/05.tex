\section{Überblick}

\begin{frame}
  {Wortbildung | Komposition}
  \onslide<+->
  \begin{itemize}[<+->]
    \item Wiederholung | statische und volatile Merkmale
    \item Wiederholung | Wortbildung und Flexion
      \Zeile
    \item Produktivität und Transparenz
    \item Köpfe und Typen von Komposita
    \item Kompositionsfugen
      \Zeile
    \item \citet[Abschnitt~8.1]{Schaefer2018b}
  \end{itemize}
\end{frame}

\section{Wortbildung}

\begin{frame}
  {Wiederholung | Statische und volatile Merkmale}
  \onslide<+->
  \begin{itemize}[<+->]
    \item Eigenschaften: "`Rotsein"' (Erdbeere), "`325m hoch"' (Eiffelturm) usw.
    \item Merkmale: \alert{\textsc{Farbe}}, \alert{\textsc{Länge}} usw.
    \item Werte:
      \begin{itemize}[<+->]
        \item \alert{\textsc{Farbe}}: \rot{\textit{rot}}, \rot{\textit{grau}}, \ldots
        \item \alert{\textsc{Länge}}: \rot{\textit{3cm}}, \rot{\textit{325m}}, \ldots
      \end{itemize}
  \end{itemize}
  \onslide<+->
  \Halbzeile 
  \begin{exe}
    \ex
    \begin{xlist}
      \ex{Haus = [\textsc{Bed}: \gruen<12->{\textbf{\textit{haus}}}, \textsc{Klasse}: \gruen<12->{\textbf{\textit{subst}}}, \textsc{Gen}: \gruen<12->{\textbf{\textit{neut}}}, \textsc{Kas}: \orongsch<13->{\textit{nom}}, \textsc{Num}: \orongsch<13->{\textit{sg}}]}
      \onslide<+->
      \ex{Haus-es = [\textsc{Bed}: \gruen<12->{\textbf{\textit{haus}}}, \textsc{Klasse}: \gruen<12->{\textbf{\textit{subst}}}, \textsc{Gen}: \gruen<12->{\textbf{\textit{neut}}}, \textsc{Kas}: \orongsch<13->{\textit{gen}}, \textsc{Num}: \orongsch<13->{\textit{sg}}]}
      \onslide<+->
      \ex{Häus-er = [\textsc{Bed}: \gruen<12->{\textbf{\textit{haus}}}, \textsc{Klasse}: \gruen<12->{\textbf{\textit{subst}}}, \textsc{Gen}: \gruen<12->{\textbf{\textit{neut}}}, \textsc{Kas}: \orongsch<13->{\textit{nom}}, \textsc{Num}: \orongsch<13->{\textit{pl}}]}
    \end{xlist}
  \end{exe}
  \Halbzeile
  \begin{itemize}[<+->]
    \item bei einem lexikalischen Wort:
      \begin{itemize}
        \item \gruen{statische Merkmale} wertestabil
        \item \orongsch{volatile Merkmale} werteverändernd im Paradigma
      \end{itemize}
  \end{itemize}
\end{frame}

\begin{frame}
  {Wiederholung | Wortbildung und Flexion}
  \onslide<+->
  \begin{exe}
    \ex
    \begin{xlist}
      \ex trocken (Adj) → \alert{Trocken}\rot{-heit} (Subst)
      \ex Kauf (Subst), Rausch (Subst) → \alert{Kauf}\rot{-rausch} (Subst)
      \ex gehen (V) → \alert{be}\rot{-gehen} (V)
    \end{xlist}
    \onslide<+->
    \ex
    \begin{xlist}
      \ex \alert{lauf}\rot{-en} (1\slash 3 Pl Prs Ind) → \alert{lauf}\rot{-e} (1 Sg Prs Ind)
      \ex \alert{Münze} (Sg) → \alert{Münze}\rot{-n} (Pl)
    \end{xlist}
  \end{exe}
  \Halbzeile
  \begin{itemize}[<+->]
    \item Wortbildung
      \begin{itemize}[<+->]
        \item statische Merkmale geändert (Wortklasse, Bedeutung)
        \item \ldots oder gelöscht (alles außer Bedeutung: Erstglied bei Komposition)
        \item \ldots oder umgebaut (Valenz von Verben beim Applikativ)
        \item \alert{produktives Erschaffen neuer lexikalischer Wörter}
      \end{itemize}
  \Halbzeile
    \item Flexion
      \begin{itemize}
        \item Änderung der Werte volatiler Merkmale
        \item typisch: Anpassung an syntaktischen Kontext
      \end{itemize}
  \end{itemize}
\end{frame}

\begin{frame}
  {Wortbildung}
  \onslide<+->
  \begin{itemize}[<+->]
    \item virtuell unbegrenzter Wortschatz
      \Zeile
    \item gut durchschaubares und \alert{gut lernbares} System\\
      \grau{trotz vieler Probleme und Einschränkungen im Detail}
      \Zeile
    \item Funktionen der Wortbildung
      \begin{itemize}
        \item Komposition | \alert{komplexe Konzepte} (\textit{Lötzinnschmelztemperatur})
        \item Konversion | \alert{Reifizierung} (z.B.\ eines Ereignisses als Objekt: \textit{der Lauf})
        \item Derivation | \alert{Modifikation von Bedeutungen} (\textit{\alert{un}schön}),\\
          \alert{Bezug auf Teilaspekte von Konzepten} (z.\,B.\ Ereigniskonzepten: \textit{Fahr\alert{er}})
      \end{itemize}
      \Halbzeile
    \item Hauptproblem der Wortbildung:\\
      \rot{Welche Bildungen sind wirklich produktiv?}
  \end{itemize}
\end{frame}

\begin{frame}
  {Relevanz von Komposition}
  \onslide<+->
  \begin{itemize}[<+->]
    \item Wortbildung als einer der Kerne der Bildungssprache
    \item kann sowohl \alert{verdichten} als auch \alert{präzisieren} \grau{\citep{Feilke2012}}
    \Halbzeile
    \item komplexe Sachverhalte \alert{optimiert} formulieren
      \begin{itemize}[<+->]
        \item möglichst kurz bzw.\ kompakt
        \item maximal verständlich (Wortbildung hochgradig etabliert\\
          im Deutschen → problemlose Verarbeitung durch Hörer*innen)
      \end{itemize}
      \Halbzeile
    \item Aber \rot{das Unterrichten von Regularitäten bzgl.\\
      der externen Funktionen ist bei Wortbildung schwierig.}
      \Halbzeile
      \begin{itemize}[<+->]
        \item "`Wenn du kommunikativ X erreichen willst,\\
          nimm eine Derivation auf \textit{-igkeit}."' \onslide<+-> \rot{Wohl kaum!}
        \item \alert{allgemeine souveräne Beherrschung des formalen Systems →\\
          globale Optimierung der Schrift- und Bildungssprache}
      \end{itemize}
  \end{itemize}
\end{frame}

\section{Komposition}

\begin{frame}
  {Beispiele für Komposition}
  \onslide<+->
  Komposition: \alert{Stamm\Sub{1} + Stamm\Sub{2} → neuer Stamm\Sub{3}}
  \Halbzeile
  \onslide<+->
  \begin{exe}
    \ex
    \begin{xlist}
      \ex{Kopf.\alert{hörer}}
      \onslide<+->
      \ex{Laut.\alert{sprecher}}
      \onslide<+->
      \ex{Kraft.\alert{werk}}
      \onslide<+->
      \ex{Lehr.\alert{veranstaltung}}
      \onslide<+->
      \ex{Rot.\alert{eiche}}
      \onslide<+->
      \ex{Lauf.\alert{schuhe}}
      \onslide<+->
      \ex{Ess.\alert{besteck}}
      \onslide<+->
      \ex{Fertig.\alert{gericht}}
      \onslide<+->
      \ex{feuer.\alert{rot}}
    \end{xlist}
  \end{exe}
\end{frame}

\begin{frame}
  {Produktivität und Transparenz}
  \onslide<+->
  \begin{itemize}[<+->]
    \item \alert{alle} Beispiele auf der vorherigen Folie: \alert{lexikalisiert}
      \begin{itemize}[<+->]
        \item vergleichsweise häufig vorkommende Wörter
        \item überwiegend spezifischere Bedeutung, als Bestandteile vermuten lassen
        \item aber: Art der Bildung erkennbar
        \item zumindest für erwachsene Sprecher*innen auch bewusst
      \end{itemize}
      \Halbzeile
    \item \alert{transparent}: Rekonstruierbarkeit der Bildung\\
      (auch bei abweichender Gesamtbedeutung)
      \Halbzeile
    \item \alert{produktiv gebildet}: Neubildung durch Sprecher*innen\\
      in einer gegebenen Situation
    \item Produktivität ist \rot{graduell} aufzufassen!
    \item \orongsch{\textit{Buchbutter}} > \textit{Batterieschublade} > \textit{Laufschuhe} > \gruen{\textit{Hundstage}}
  \end{itemize}
\end{frame}

\begin{frame}[fragile,label=hierarchie]
  {Rekursion}
  \onslide<+->
  \onslide<+->
  \begin{center}
    \scalebox{0.7}{
      \begin{forest}
        [Bushaltestellenunterstandsreparatur
          [Bushaltestellenunterstand
            [Bushaltestelle
              [Bus]
              [Haltestelle
                [halten]
                [Stelle]
              ]
            ]
            [Unterstand
              [unter]
              [Stand]
            ]
          ]
          [Reparatur]
        ]
      \end{forest}
    }
  \end{center}
  \begin{itemize}[<+->]
    \item Wortbildung: immer \alert{binär}, also \alert{Wort+Wort} (nicht \rot{Wort+Wort+Wort} usw.)
      \Viertelzeile
    \item \alert{hierarchische Strukturbildung} durch wiederholte lineare Anfügung
      \Viertelzeile
    \item Rekursion allgemein: \alert{Eine Verknüpfung hat als Ergebnis\\
      eine Einheit, die wieder auf dieselbe Art verknüpft werden kann.}
    \item Rekursion in Linguistik: immer eingeschränkt, nicht "`endlos"'
  \end{itemize}
\end{frame}

\begin{frame}
  {Köpfe}
  \onslide<2->
  \begin{exe}
    \ex
    \begin{xlist}
      \ex \orongsch<19->{Laut}.\alert<10->{sprecher} \onslide<19->{\orongsch{(\textit{laut} verliert Wortklasse, \dots)}}
      \onslide<3->
      \ex \orongsch<20->{Kraft}.\alert<11->{werk} \onslide<20->{\orongsch{(\textit{Kraft} verliert Wortklasse, Genus, \dots)}}
      \onslide<4->
      \ex \orongsch<21->{Lauf}.\alert<12->{schuhe} \onslide<21->{\orongsch{(\textit{laufen} verliert Wortklasse? Genus? \dots)}}
      \onslide<5->
      \ex \orongsch<22->{Ess}.\alert<13->{besteck} \onslide<22->{\orongsch{(\textit{essen} verliert Wortklasse, \dots)}}
      \onslide<6->
      \ex \orongsch<23->{feuer}.\alert<14->{rot} \onslide<23->{\orongsch{(\textit{Feuer} verliert Wortklasse, \dots)}}
    \end{xlist}
  \end{exe}
  \onslide<7->
  \begin{itemize}
    \item \alert{Kopf}:
      \begin{itemize}
          \onslide<8->
        \item steht immer rechts
          \onslide<9->
        \item bestimmt alle grammatischen Merkmale des Kompositums
      \end{itemize}
      \Halbzeile
      \onslide<15->
    \item \orongsch{Nicht-Kopf}
      \begin{itemize}
          \onslide<16->
        \item immer links
          \onslide<17->
        \item verliert alle grammatischen Merkmale
          \onslide<18->
        \item Bedeutung geht in Gesamtbedeutung ein
      \end{itemize}
  \end{itemize}
\end{frame}

\begin{frame}
  {Relevante Kompositionstypen: Determinativkomposita}
  \onslide<+->
  Determinativkomposita: \textit{Schulheft}, \textit{Regalbrett} usw.
  \Halbzeile
  \begin{itemize}[<+->]
    \item Kopf--Kern-Test:
      \begin{itemize}[<+->]
        \item Ein Schulheft ist ein Heft. \Ck
        \item Ein Regalbrett ist ein Brett. \Ck
      \end{itemize}
    \item Nicht-Kopf--Kern-Test:
      \begin{itemize}[<+->]
        \item Ein Schulheft ist eine Schule. \Fl
        \item Ein Regalbrett ist ein Regal. \Fl
      \end{itemize}
      \Halbzeile
    \item Rektionstest:
      \begin{itemize}[<+->]
        \item \rot{Bei einem Schulheft wird eine Schule geheftet\slash verheftet\slash beheftet... \Fl}
        \item \rot{Bei einem Regalbrett wird ein Regal gebrettert\slash\dots \Fl}
      \end{itemize}
  \end{itemize}
\end{frame}


\begin{frame}
  {Relevante Kompositionstypen: Rektionskomposita}
  \onslide<+->
  Rektionskomposita: \textit{Hemdenwäsche}, \textit{Geldfälschung} usw.
  \Halbzeile
  \begin{itemize}[<+->]
    \item Kopf--Kern-Test:
      \begin{itemize}[<+->]
        \item Eine Hemdenwäsche ist eine Wäsche. \Ck
        \item Eine Geldfälschung ist eine Fälschung. \Ck
      \end{itemize}
    \item Nicht-Kopf--Kern-Test:
      \begin{itemize}[<+->]
        \item Eine Hemdenwäsche ist ein Hemd. \Fl
        \item Eine Geldfälschung ist Geld. \Fl
      \end{itemize}
      \Halbzeile
    \item Rektionstest:
      \begin{itemize}[<+->]
        \item \alert{Bei einer Hemdenwäsche werden Hemden gewaschen. \Ck}
        \item \alert{Bei einer Geldfälschung wird Geld gefälscht. \Ck}
      \end{itemize}
      \Halbzeile
    \item Kopf: prototypischerweise von einem Verb abgeleitet
    \item Nicht-Kopf zu Kopf wie Objekt zu Verb
  \end{itemize}
\end{frame}

\begin{frame}
  {Kompositionsfugen bei Substantiv-Substantiv-Komposita}
  \onslide<+->
  \onslide<+->
  \begin{center}
    \scalebox{1}{
      \begin{tabular}{llrr}
        \toprule
        Fuge          & Beispiel                        & Komposita \% & Erstglieder \% \\
        \midrule                                                                                                    
        $\varnothing$ & \textit{Garten.tür}             & 60.25        & 41.77          \\ 
        -(e)s         & \textit{Gelegenheit-s.dieb}     & 23.69        & 45.74          \\ 
        -n            & \textit{Katze-n.pfote}          & 10.38        &  5.29          \\ 
        -en           & \textit{Frau-en.stimme}         &  3.02        &  4.19          \\ 
        *e            & \textit{Kirsch.kuchen}          &  0.78        &  0.20          \\ 
        -e            & \textit{Geschenk-e.laden}       &  0.71        &  1.90          \\ 
        -er           & \textit{Kind-er.buch}           &  0.38        &  0.07          \\ 
        \char`~er     & \textit{Büch-er.regal}          &  0.37        &  0.11          \\ 
        \char`~e      & \textit{Händ-e.druck}           &  0.22        &  0.63          \\ 
        -ns           & \textit{Name-ns.schutz}         &  0.13        &  0.04          \\ 
        \char`~       & \textit{Mütter.zentrum}        &  0.05        &  0.06           \\ 
        -ens          & \textit{Herz-ens.angelegenheit} &  0.03        &  0.01          \\ 
        \bottomrule
      \end{tabular}
    }\\
    \Halbzeile
    \footnotesize{(aus: \citealt{SchaeferPankratz2018})}
  \end{center}
\end{frame}

\begin{frame}
  {Steuerung der Fugen durch Erstglied}
  \onslide<+->
  \begin{itemize}[<+->]
    \item Wörter mit s-Plural (\textit{Kaffees}, \textit{Kameras}) \rot{niemals mit s-Fuge}
      \Halbzeile
    \item \alert{derivierte} Substantive (meist Abstrakta) (\textit{-heit}, \textit{-keit}, \textit{-tum}):\\
      \alert{prototypisch s-Fuge}
      \begin{itemize}[<+->]
        \item sehr viele Feminina, Fuge nicht paradigmatisch (= keine Flexionsform)
      \end{itemize}
      \Halbzeile
    \item starke\slash gemischte Maskulina: manchmal -(\textit{e})\textit{s}
      \begin{itemize}[<+->]
        \item Genitiv? Welche Funktion sollte ein Genitiv im Kompositum haben?
        \item Lassen sich die Komposita mit s-Fuge mit Genitiv umformulieren?
        \item \textit{Freundeskreis → \rot{*Kreis des Freundes}}
        \item \textit{Geschlechtsverkehr → \rot{*Verkehr des Geschlechts}}
        \item \textit{Berufstätigkeit → \rot{*Tätigkeit des Berufs}}
        \item \textit{Auslandsaufenthalt → \rot{*Aufenthalt des Auslands}}
      \end{itemize}
    \Halbzeile
  \item die s-Fugen an \alert{Feminina} sowieso nicht als Genitiv möglich:
      \begin{itemize}
        \item \textit{Gelegenheitsdieb} → \rot{*\textit{Dieb der Gelegenheits}}
      \end{itemize}
  \end{itemize}
\end{frame}

\section{Übung}

\begin{frame}
  {Komposita im Text}
  \begin{enumerate}
    \item Suchen Sie im gegebenen Text \alert{zehn Komposita}.
    \item Zeichnen Sie für jedes einen \alert{Baum} wie auf Folie \ref{hierarchie}.
    \item Unterstreichen Sie bei jeder Verzweigung im Baum den \alert{Kopf}.
    \item Was ist der Kopf, der alle grammatischen Merkmale\\
      des Kompositums bestimmt?
    \item Bestimmen Sie alle \alert{Fugen} und entscheiden Sie,\\
      ob sie \alert{paradigmatisch} sind.
    \item Überlegen Sie, ob die einzelnen Kompositionen\\
      \alert{produktiv} oder \alert{transparent} sind.\\
      \grau{Dazu gibt es oft keine endgültige Lösung.}
  \end{enumerate}
\end{frame}

\section{Ausblick}

\begin{frame}
  {Andere Wortbildungsmuster}
  \onslide<+->
  \begin{itemize}[<+->]
    \item \alert{Konversion} | Stamm\Sub{1} → Stamm\Sub{2} \\ 
      \textit{laufen} → (\textit{der}) \textit{Lauf}
      \Zeile
    \item \alert{Derivation} | Stamm\Sub{1} + Affix → Stamm\Sub{2}\\
      \textit{schön} → (\textit{die}) \textit{Schönheit}
      \Halbzeile
    \item Typische Anwendungsbereiche für \alert{Präfigierung} und\\
      \alert{Suffigierung} im Deutschen
  \end{itemize}
\end{frame}
